\documentclass[letterpaper,10pt,titlepage]{article}                                     

% TODO:
% Remove some Achievements and awards, except for a couple major ones?
%    this section would be better filled with NSF GRFP if I get it.

% Referenced code:
% uses R. Geirhos's method of putting publications into a CV, as shown in their GitHub repository
% https://github.com/rgeirhos/academic-cv-publications

\usepackage{amssymb}                                         
\usepackage{amsmath}                                         
\usepackage{amsthm}                                          

\usepackage{alltt}                                           
\usepackage{float}
\usepackage{color}

\usepackage{url}

\usepackage{balance}
\usepackage[TABBOTCAP, tight]{subfigure}
\usepackage{enumitem}

\usepackage{pstricks, pst-node}

\usepackage[margin=1in]{geometry}

\usepackage{tabularx}
\usepackage{titlesec}
\usepackage{verbatim}

\usepackage{bibentry}
\makeatletter\let\saved@bibitem\@bibitem\makeatother
\usepackage[colorlinks=true]{hyperref}
\makeatletter\let\@bibitem\saved@bibitem\makeatother
\usepackage{hanging}
\newcommand\publication[1]{%
	\smallskip\par\hangpara{1.5em}{1}\bibentry{#1}\smallskip
}


%% The following metadata will show up in the PDF properties
\hypersetup{
  colorlinks = true,
  urlcolor = black,
  pdfauthor = {Connor Yates},
  pdfkeywords = {resume},
  pdftitle = {Connor Yates Resume},
  pdfsubject = {Resume},
  pdfpagemode = UseNone
}

% Use a variable to set a standard column width across all two-column elements in the document
\newcommand{\leftW}{0.32\textwidth}
\parindent = 0.0 in
%\parskip = 0.2 in

%Modify the \section command
\titleformat{\section}
	{\normalfont\Large}{\thesection}{1em}{}[]%{\titlerule[0.6pt]}% <-- Put inside [] to bring back section lines
\titlespacing{\section}{0pt}{1pt}{3pt}

\begin{document}

\pagestyle{empty}%                                                      Get rid of page numbers and such
\par{\centering{
	{\LARGE \textbf{Connor Yates}}\\
	\vspace{0.3em}
	Ph.D. Student, Robotics\\
	Oregon State University\\
	204 Rogers hall, Corvallis, OR 97331-6770\\
	(541) 602-7403\\
	\url{yatesco@oregonstate.edu} \\
	\url{http://people.oregonstate.edu/\~yatesco}
	\rule{\linewidth}{0.4pt}
	}\par}

\section*{\textsc{Research Focus}}
My interests are in artificial intelligence methods for collaboration and coordination in teams of autonomous agents. I am investigating multi-reward reinforcement learning structures to train teams to operate in tasks with fuzzy concepts of success. 
I am also interested in how teams of agents synthesize explanations for human observers.

\section*{\textsc{Education}}
% College Description
\begin{tabularx}{\textwidth}{p{\leftW}X}
	Ph.D. Student in Robotics & \textbf{Oregon State University}, Corvallis OR 97331\\

	\mbox{2017 - 2022} & Advised by Kagan Tumer. Researching methods for introspection and coordination in multiagent systems.\\
	 &\\
\end{tabularx}
\begin{tabularx}{\textwidth}{p{\leftW}X}
	Honors B.S., \textit{magna cum laude} & \textbf{Oregon State University}, Corvallis OR 97331\\
	2017 & Honors Bachelor of Science in Computer Science, College of Engineering and Honors College, 3.84 GPA \\
\end{tabularx}

\section*{\textsc{Skills and Interests}}
\begin{itemize} \itemsep1pt \parskip0pt \parsep0pt
\item Interested in solving distributed optimization, multi-robot control problems, and robotics problems in the real world
\item Research involving neural networks, machine learning, reinforcement learning, and evolutionary learning techniques
\item In-depth knowledge and experience with GNU/Linux, Python, C/C++, \LaTeX, Git, and ROS
\end{itemize}


\section*{\textsc{Teaching}}
\begin{tabularx}{\linewidth}{p{\leftW}X}
	\textbf{\mbox{Computer Programming for} \mbox{Mechanical Systems}} & Teaching Assistant, Winter 2018\\
	ME 499/599 & Taught basics of the Python programming language and how it can be used for analysis and control of mechanical systems. \\
	\\
	\textbf{Intelligent Robotics}&Teaching Assistant, Fall 2017, Fall 2018\\
	ROB 456 & Taught modern algorithms and tools for robotics development and deployment like Bayes rule, SLAM, A*, and ROS. Helped debug programs interfacing with ROS. \\
	\\
	\textbf{\mbox{Introduction to} \mbox{Computer Science}} & Teaching Assistant, Fall '14, Winter '15, Spring '15, Fall '15, Winter '16, Spring '16\\
	CS 160, CS 161, CS 162 &  Lead laboratory sections in teams of 3 TA's. Taught fundamental programming concepts like memory and pointers, functions, and recursion in Python, C, and C++.
\end{tabularx}


\section*{\textsc{Professional Experience}}

\begin{tabularx}{\linewidth}{Xr}
	\textbf{OSU Robotics Graduate Student Association} & Corvallis, OR\\
\end{tabularx}
\begin{tabularx}{\linewidth}{p{\leftW}X}
Faculty Relations Officer          & June 2018 -- June 2019\\
\end{tabularx}
\begin{itemize} \itemsep1pt \parskip0pt \parsep0pt
\item Communicate with faculty members to raise awareness of RGSA events and issues in the department
\item Attend departmental faculty meetings on behalf of RGSA
\item Coordinate departmental events with graduate students and faculty members
\end{itemize}

\begin{tabularx}{\linewidth}{Xr}
	\textbf{OSU Autonomous Agents and Distributed Intelligence Laboratory} & Corvallis, OR\\
\end{tabularx}
\begin{tabularx}{\linewidth}{p{\leftW}X}
Undergraduate Research Assistant          & February 2015 -- June 2017\\
\end{tabularx}
\begin{itemize} \itemsep1pt \parskip0pt \parsep0pt
\item Collaborated with other student researchers from around the country on novel research on incorporating others' intent into the perceived decision-making state
\item Researched methods for reinforcement learning in tightly coupled multiagent domains
\item Created optimal control policies through a combination of standard neuro-evolutionary methods and hierarchical decision making
\end{itemize}

\begin{tabularx}{\linewidth}{Xr}
	\textbf{Chick Tech} & Corvallis, OR\\
\end{tabularx}
\begin{tabularx}{\linewidth}{p{\leftW}X}
Workshop Lead Volunteer          &  August 2014, September 2015\\
\end{tabularx}
\begin{itemize} \itemsep1pt \parskip0pt \parsep0pt
\item Developed a curriculum for a weekend workshop on game development with a team of volunteers
\item Taught the weekend workshop to high school girls to encourage their interest in STEM fields
\end{itemize}

\begin{tabularx}{\linewidth}{Xr}
	\textbf{OSU CARVE Lab} & Corvallis, OR\\
\end{tabularx}
\begin{tabularx}{\linewidth}{p{\leftW}X}
	Simulation Developer   & March 2014 -- September 2014\\
\end{tabularx}
\begin{itemize} \itemsep1pt \parskip0pt \parsep0pt
\item Worked with a team of psychologists to create virtually simulated testing environments for experiments
\item Designed, programmed, and debugged a testing environment to the standards of the experiment design
\item Created virtual testing environments using Python and the Vizard Virtual Reality libraries for Spectroscopic Head-Mounted Displays Simulations
\end{itemize}

\begin{comment} % COMMENTED OUT CROSSROADS
\begin{tabularx}{\linewidth}{Xr}
\textbf{\textit{Crossroads Carnegie Art Center}} & \textbf{Volunteer}\\
\textbf{Technical Support, Web Master}           & June 2013 -- August 2013\\
Baker City, Oregon & \\
\end{tabularx}

\begin{itemize} \itemsep1pt \parskip0pt \parsep0pt
\item Created digital backups of previous years tax information
\item Transferred customer data to a new cloud database for the Center
\item Maintained custom PHP/XHTML website to specifications
\item Provided consultation on creation of requirements for a new website
\end{itemize}
\end{comment}

\section*{Public Talks and Outreach}
\begin{itemize}
	\item Linux Users Group and ACM Chapter, Oregon State University. ``Open Source Robotics,'' February 2018
	\item Inspiration Dissemination (Radio, KBVR), Oregon State University. ``How many robots does it take to screw in a light bulb?'' February 2018. 
\end{itemize}


\section*{\textsc{Publications}}
\nobibliography{publications}
\bibliographystyle{acm}
\publication{Chung2018}
\publication{Khadka:2018:MMF:3237383.3238043}
\publication{klinkhammer-18}
\publication{Khadka:2019}

\section*{\textsc{Achievements and Awards}}
\begin{tabularx}{\linewidth}{>{\centering\arraybackslash}X|>{\centering\arraybackslash}X}
%OSU Honor Roll Fall Term 2013, Winter Term 2014, Spring Term 2014 & Leo Adler Scholarship \\ %Move the leo to the bottom for now, and consider what to do with the honor roll
%Charis Initiative Scholarship & Rhodes Brothers Scholarship \\
%Alan McCullough Scholarship & OSU Computer Science Scholarship \\
OSU Academic Achievement Award & OSU Dean's Engineering Scholarship \\
Leo Adler Scholarship & Oregon State University Honor Roll\\
\end{tabularx}


\end{document}


